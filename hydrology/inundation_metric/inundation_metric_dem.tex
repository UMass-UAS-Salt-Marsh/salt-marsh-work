% Options for packages loaded elsewhere
\PassOptionsToPackage{unicode}{hyperref}
\PassOptionsToPackage{hyphens}{url}
%
\documentclass[
]{article}
\usepackage{amsmath,amssymb}
\usepackage{iftex}
\ifPDFTeX
  \usepackage[T1]{fontenc}
  \usepackage[utf8]{inputenc}
  \usepackage{textcomp} % provide euro and other symbols
\else % if luatex or xetex
  \usepackage{unicode-math} % this also loads fontspec
  \defaultfontfeatures{Scale=MatchLowercase}
  \defaultfontfeatures[\rmfamily]{Ligatures=TeX,Scale=1}
\fi
\usepackage{lmodern}
\ifPDFTeX\else
  % xetex/luatex font selection
\fi
% Use upquote if available, for straight quotes in verbatim environments
\IfFileExists{upquote.sty}{\usepackage{upquote}}{}
\IfFileExists{microtype.sty}{% use microtype if available
  \usepackage[]{microtype}
  \UseMicrotypeSet[protrusion]{basicmath} % disable protrusion for tt fonts
}{}
\makeatletter
\@ifundefined{KOMAClassName}{% if non-KOMA class
  \IfFileExists{parskip.sty}{%
    \usepackage{parskip}
  }{% else
    \setlength{\parindent}{0pt}
    \setlength{\parskip}{6pt plus 2pt minus 1pt}}
}{% if KOMA class
  \KOMAoptions{parskip=half}}
\makeatother
\usepackage{xcolor}
\usepackage[margin=1in]{geometry}
\usepackage{color}
\usepackage{fancyvrb}
\newcommand{\VerbBar}{|}
\newcommand{\VERB}{\Verb[commandchars=\\\{\}]}
\DefineVerbatimEnvironment{Highlighting}{Verbatim}{commandchars=\\\{\}}
% Add ',fontsize=\small' for more characters per line
\usepackage{framed}
\definecolor{shadecolor}{RGB}{248,248,248}
\newenvironment{Shaded}{\begin{snugshade}}{\end{snugshade}}
\newcommand{\AlertTok}[1]{\textcolor[rgb]{0.94,0.16,0.16}{#1}}
\newcommand{\AnnotationTok}[1]{\textcolor[rgb]{0.56,0.35,0.01}{\textbf{\textit{#1}}}}
\newcommand{\AttributeTok}[1]{\textcolor[rgb]{0.13,0.29,0.53}{#1}}
\newcommand{\BaseNTok}[1]{\textcolor[rgb]{0.00,0.00,0.81}{#1}}
\newcommand{\BuiltInTok}[1]{#1}
\newcommand{\CharTok}[1]{\textcolor[rgb]{0.31,0.60,0.02}{#1}}
\newcommand{\CommentTok}[1]{\textcolor[rgb]{0.56,0.35,0.01}{\textit{#1}}}
\newcommand{\CommentVarTok}[1]{\textcolor[rgb]{0.56,0.35,0.01}{\textbf{\textit{#1}}}}
\newcommand{\ConstantTok}[1]{\textcolor[rgb]{0.56,0.35,0.01}{#1}}
\newcommand{\ControlFlowTok}[1]{\textcolor[rgb]{0.13,0.29,0.53}{\textbf{#1}}}
\newcommand{\DataTypeTok}[1]{\textcolor[rgb]{0.13,0.29,0.53}{#1}}
\newcommand{\DecValTok}[1]{\textcolor[rgb]{0.00,0.00,0.81}{#1}}
\newcommand{\DocumentationTok}[1]{\textcolor[rgb]{0.56,0.35,0.01}{\textbf{\textit{#1}}}}
\newcommand{\ErrorTok}[1]{\textcolor[rgb]{0.64,0.00,0.00}{\textbf{#1}}}
\newcommand{\ExtensionTok}[1]{#1}
\newcommand{\FloatTok}[1]{\textcolor[rgb]{0.00,0.00,0.81}{#1}}
\newcommand{\FunctionTok}[1]{\textcolor[rgb]{0.13,0.29,0.53}{\textbf{#1}}}
\newcommand{\ImportTok}[1]{#1}
\newcommand{\InformationTok}[1]{\textcolor[rgb]{0.56,0.35,0.01}{\textbf{\textit{#1}}}}
\newcommand{\KeywordTok}[1]{\textcolor[rgb]{0.13,0.29,0.53}{\textbf{#1}}}
\newcommand{\NormalTok}[1]{#1}
\newcommand{\OperatorTok}[1]{\textcolor[rgb]{0.81,0.36,0.00}{\textbf{#1}}}
\newcommand{\OtherTok}[1]{\textcolor[rgb]{0.56,0.35,0.01}{#1}}
\newcommand{\PreprocessorTok}[1]{\textcolor[rgb]{0.56,0.35,0.01}{\textit{#1}}}
\newcommand{\RegionMarkerTok}[1]{#1}
\newcommand{\SpecialCharTok}[1]{\textcolor[rgb]{0.81,0.36,0.00}{\textbf{#1}}}
\newcommand{\SpecialStringTok}[1]{\textcolor[rgb]{0.31,0.60,0.02}{#1}}
\newcommand{\StringTok}[1]{\textcolor[rgb]{0.31,0.60,0.02}{#1}}
\newcommand{\VariableTok}[1]{\textcolor[rgb]{0.00,0.00,0.00}{#1}}
\newcommand{\VerbatimStringTok}[1]{\textcolor[rgb]{0.31,0.60,0.02}{#1}}
\newcommand{\WarningTok}[1]{\textcolor[rgb]{0.56,0.35,0.01}{\textbf{\textit{#1}}}}
\usepackage{graphicx}
\makeatletter
\def\maxwidth{\ifdim\Gin@nat@width>\linewidth\linewidth\else\Gin@nat@width\fi}
\def\maxheight{\ifdim\Gin@nat@height>\textheight\textheight\else\Gin@nat@height\fi}
\makeatother
% Scale images if necessary, so that they will not overflow the page
% margins by default, and it is still possible to overwrite the defaults
% using explicit options in \includegraphics[width, height, ...]{}
\setkeys{Gin}{width=\maxwidth,height=\maxheight,keepaspectratio}
% Set default figure placement to htbp
\makeatletter
\def\fps@figure{htbp}
\makeatother
\setlength{\emergencystretch}{3em} % prevent overfull lines
\providecommand{\tightlist}{%
  \setlength{\itemsep}{0pt}\setlength{\parskip}{0pt}}
\setcounter{secnumdepth}{-\maxdimen} % remove section numbering
\ifLuaTeX
  \usepackage{selnolig}  % disable illegal ligatures
\fi
\usepackage{bookmark}
\IfFileExists{xurl.sty}{\usepackage{xurl}}{} % add URL line breaks if available
\urlstyle{same}
\hypersetup{
  pdftitle={inundation\_metric\_dem},
  pdfauthor={emily},
  hidelinks,
  pdfcreator={LaTeX via pandoc}}

\title{inundation\_metric\_dem}
\author{emily}
\date{2025-06-24}

\begin{document}
\maketitle

\begin{Shaded}
\begin{Highlighting}[]
\CommentTok{\#this file is found in "/Data\_AllSites/Derived Raster/Inundation Maps from Elevation Models/Prediction Raster on Inundation Metrics" which is found in the drive folder}

\NormalTok{four\_sites }\OtherTok{\textless{}{-}} \FunctionTok{readRDS}\NormalTok{(}\StringTok{"/Users/emily/Library/CloudStorage/GoogleDrive{-}ekmiller@umass.edu/.shortcut{-}targets{-}by{-}id/0B6{-}MI{-}dco6FLWkZmTDZ4MFhRU1k/7. SaltMUAS\_share/Data\_AllSites/Derived Raster/Inundation Maps from Elevation Models/Prediction Raster on Inundation Metrics/four\_sites.Rds"}\NormalTok{)}
\end{Highlighting}
\end{Shaded}

\begin{Shaded}
\begin{Highlighting}[]
\CommentTok{\#used to create the DEM as an object. Look within the "Data\_AllSites/Original Raster/UAS LiDAR DTM" to find DTMs for each site}

\NormalTok{dtm }\OtherTok{\textless{}{-}}\NormalTok{ terra}\SpecialCharTok{::}\FunctionTok{rast}\NormalTok{(}\StringTok{"/Users/emily/Library/CloudStorage/GoogleDrive{-}ekmiller@umass.edu/.shortcut{-}targets{-}by{-}id/0B6{-}MI{-}dco6FLWkZmTDZ4MFhRU1k/7. SaltMUAS\_share/Data\_AllSites/Original Raster/UAS LiDAR DTM/WES/WES\_20May2022\_CSF2012\_DTM\_clipped\_v2.tif"}\NormalTok{)}
\end{Highlighting}
\end{Shaded}

\begin{Shaded}
\begin{Highlighting}[]
\CommentTok{\#The next three chunks are all the same with the linear model being different each time due to the change of the metric. }

\CommentTok{\#filter data for site where you pulled your desired dem from}
\NormalTok{data}\OtherTok{\textless{}{-}}\NormalTok{four\_sites }\SpecialCharTok{|\textgreater{}}
   \FunctionTok{filter}\NormalTok{(Site }\SpecialCharTok{==} \StringTok{"WES"}\NormalTok{)}

\CommentTok{\#metric linear model for raster creation using Proportional\_Time\_Inundated as metric, any calculated metric can be used in place of Proportional\_Time\_Inundated}
\NormalTok{prop\_time\_model }\OtherTok{\textless{}{-}} \FunctionTok{lm}\NormalTok{(Proportion\_Time\_Inundated}\SpecialCharTok{\textasciitilde{}}\NormalTok{Elevation, }\AttributeTok{data =}\NormalTok{ data)    }
\FunctionTok{names}\NormalTok{(dtm) }\OtherTok{\textless{}{-}} \StringTok{"Elevation"}

\CommentTok{\#raster prediction creation}
\NormalTok{pti }\OtherTok{\textless{}{-}} \FunctionTok{predict}\NormalTok{(dtm, prop\_time\_model)}
\end{Highlighting}
\end{Shaded}

\begin{Shaded}
\begin{Highlighting}[]
\NormalTok{data}\OtherTok{\textless{}{-}}\NormalTok{four\_sites }\SpecialCharTok{|\textgreater{}}
   \FunctionTok{filter}\NormalTok{(Site }\SpecialCharTok{==} \StringTok{"WES"}\NormalTok{)}

\NormalTok{med\_time\_model }\OtherTok{\textless{}{-}} \FunctionTok{lm}\NormalTok{(Median\_Time\_Inundated}\SpecialCharTok{\textasciitilde{}}\NormalTok{Elevation, }\AttributeTok{data =}\NormalTok{ data)}
\FunctionTok{names}\NormalTok{(dtm) }\OtherTok{\textless{}{-}} \StringTok{"Elevation"} 

\NormalTok{pti1 }\OtherTok{\textless{}{-}} \FunctionTok{predict}\NormalTok{(dtm, med\_time\_model)}
\end{Highlighting}
\end{Shaded}

\begin{Shaded}
\begin{Highlighting}[]
\NormalTok{data}\OtherTok{\textless{}{-}}\NormalTok{four\_sites }\SpecialCharTok{|\textgreater{}}
   \FunctionTok{filter}\NormalTok{(Site }\SpecialCharTok{==} \StringTok{"WES"}\NormalTok{)}

\NormalTok{depth\_95\_model }\OtherTok{\textless{}{-}} \FunctionTok{lm}\NormalTok{(Percentile\_95\_Depth}\SpecialCharTok{\textasciitilde{}}\NormalTok{Elevation, }\AttributeTok{data =}\NormalTok{ data)}
\FunctionTok{names}\NormalTok{(dtm) }\OtherTok{\textless{}{-}} \StringTok{"Elevation"} 

\NormalTok{pti2 }\OtherTok{\textless{}{-}} \FunctionTok{predict}\NormalTok{(dtm, depth\_95\_model)}
\end{Highlighting}
\end{Shaded}

\begin{Shaded}
\begin{Highlighting}[]
\CommentTok{\#plots used to visualize the prediction raster}

\FunctionTok{plot}\NormalTok{(pti,}
     \AttributeTok{box =} \ConstantTok{FALSE}\NormalTok{,}
     \AttributeTok{axes =} \ConstantTok{FALSE}\NormalTok{,}
     \AttributeTok{col =} \FunctionTok{colorRampPalette}\NormalTok{(}\FunctionTok{c}\NormalTok{(}\StringTok{"red"}\NormalTok{, }\StringTok{"green"}\NormalTok{, }\StringTok{"blue"}\NormalTok{))(}\DecValTok{100}\NormalTok{),}
     \AttributeTok{main =} \StringTok{"Proportion\_Time\_Inundated"}\NormalTok{)}
\end{Highlighting}
\end{Shaded}

\includegraphics{inundation_metric_dem_files/figure-latex/visual plots-1.pdf}

\begin{Shaded}
\begin{Highlighting}[]
\FunctionTok{plot}\NormalTok{(pti1,}
     \AttributeTok{box =} \ConstantTok{FALSE}\NormalTok{,}
     \AttributeTok{axes =} \ConstantTok{FALSE}\NormalTok{,}
     \AttributeTok{col =} \FunctionTok{colorRampPalette}\NormalTok{(}\FunctionTok{c}\NormalTok{(}\StringTok{"red"}\NormalTok{, }\StringTok{"green"}\NormalTok{, }\StringTok{"blue"}\NormalTok{))(}\DecValTok{100}\NormalTok{),}
     \AttributeTok{main =} \StringTok{"Median\_Time\_Inundated"}\NormalTok{)}
\end{Highlighting}
\end{Shaded}

\includegraphics{inundation_metric_dem_files/figure-latex/visual plots-2.pdf}

\begin{Shaded}
\begin{Highlighting}[]
\FunctionTok{plot}\NormalTok{(pti2,}
     \AttributeTok{box =} \ConstantTok{FALSE}\NormalTok{,}
     \AttributeTok{axes =} \ConstantTok{FALSE}\NormalTok{,}
     \AttributeTok{col =} \FunctionTok{colorRampPalette}\NormalTok{(}\FunctionTok{c}\NormalTok{(}\StringTok{"red"}\NormalTok{, }\StringTok{"green"}\NormalTok{, }\StringTok{"blue"}\NormalTok{))(}\DecValTok{100}\NormalTok{), }
     \AttributeTok{main =} \StringTok{"Percentile\_95\_Depth"}\NormalTok{)}
\end{Highlighting}
\end{Shaded}

\includegraphics{inundation_metric_dem_files/figure-latex/visual plots-3.pdf}

\begin{Shaded}
\begin{Highlighting}[]
\CommentTok{\#This is used to write out your created raster so it can be downloaded to your chosen folder for the raster}

\CommentTok{\#Example: out\_tif\_metric \textless{}{-} writeRaster(raster, "path to where you want the file/name\_of\_file.tif")}

\NormalTok{out\_tif\_prop\_time }\OtherTok{\textless{}{-}} \FunctionTok{writeRaster}\NormalTok{(pti, }\StringTok{"/Users/emily/Library/CloudStorage/OneDrive{-}UniversityofMassachusetts/salt\_marsh\_data/Testing repo/salt\_marsh\_work/hydrology/Data/WES/WES\_output\_raster\_Proportion\_Time\_Inundated.tif"}\NormalTok{, }\AttributeTok{overwrite =} \ConstantTok{TRUE}\NormalTok{)}

\NormalTok{out\_tif\_med\_time }\OtherTok{\textless{}{-}} \FunctionTok{writeRaster}\NormalTok{(pti1, }\StringTok{"/Users/emily/Library/CloudStorage/OneDrive{-}UniversityofMassachusetts/salt\_marsh\_data/Testing repo/salt\_marsh\_work/hydrology/Data/WES/WES\_output\_raster\_Median\_Time\_Inundated.tif"}\NormalTok{, }\AttributeTok{overwrite =} \ConstantTok{TRUE}\NormalTok{)}

\NormalTok{out\_tif\_95\_depth }\OtherTok{\textless{}{-}} \FunctionTok{writeRaster}\NormalTok{(pti2, }\StringTok{"/Users/emily/Library/CloudStorage/OneDrive{-}UniversityofMassachusetts/salt\_marsh\_data/Testing repo/salt\_marsh\_work/hydrology/Data/WES/WES\_output\_raster\_Percentile\_95\_Depth.tif"}\NormalTok{, }\AttributeTok{overwrite =} \ConstantTok{TRUE}\NormalTok{)}
\end{Highlighting}
\end{Shaded}


\end{document}
